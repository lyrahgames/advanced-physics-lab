% documentclass: article used for scientific journals, short reports, program documentation, etc
% options: fontsize 11, generate document for double sided printing, a4-paper
\documentclass[11pt, twoside, a4paper]{article}

% package for changing page layout
\usepackage{geometry}
\geometry{a4paper, lmargin=40mm, rmargin=45mm, tmargin=40mm, bmargin=45mm}
% set indentation
\setlength{\parindent}{1em}
% set factor for line spacing
% \linespread{1.0}\selectfont
% set (dynamic) additional line spacing
% \setlength{\parskip}{1ex plus 0.5ex minus 0.3ex}

% rigorous formatting (not too much hyphens)
% \fussy
% \sloppy

% package for changing page layout (used to indent whole paragraphs with adjustwidth)
\usepackage{changepage}

% input encoding for special characters (e.g. ä,ü,ö,ß), only for non english text
% options: utf8 as encoding standard, latin1
\usepackage[utf8]{inputenc}
% package for font encoding
\usepackage[T1]{fontenc}
% package for changing used language (especially for more than one language)
% options: ngerman (new spelling) or default: english
\usepackage[ngerman]{babel}
% package for times font
% \usepackage{times}
% package for latin modern fonts
\usepackage{lmodern}

% package for math symbols, functions and environments from ams(american mathematical society)
\usepackage{amsmath}
% package for extended symbols from ams
\usepackage{amssymb}
% package for math black board symbols (e.g. R,Q,Z,...)
\usepackage{bbm}
\usepackage{mathrsfs}
% package for extended symbols from stmaryrd(st mary road)
\usepackage{stmaryrd}

% package for including extern graphics plus scaling and rotating
\usepackage{graphicx}
%package for positioning figures
\usepackage{float}
% package for changing color of font and paper
% options: using names of default colors (e.g red, black)
% \usepackage[usenames]{color}
\usepackage[dvipsnames]{xcolor}
\definecolor{shadecolor}{gray}{0.9}
% package for customising captions
\usepackage[footnotesize, hang]{caption}
% package for customising enumerations (e.g. axioms)
\usepackage{enumitem}
% calc package reimplements \setcounter, \addtocounter, \setlength and \addtolength: commands now accept an infix notation expression
\usepackage{calc}
% package for creating framed, shaded, or differently highlighted regions that can break across pages; environments: framed, oframed, shaded, shaded*, snugshade, snugshade*, leftbar, titled-frame
\usepackage{framed}
% package for creating custom "list of"
% options: titles: do not intefere with standard headings for "list of"
\usepackage[titles]{tocloft}


% change enumeration style of equations
% \renewcommand\theequation{\thesection.\arabic{equation}}

% init list of math for definitions and theorems
\newcommand{\listofmathcall}{Verzeichnis der Definitionen und Sätze}
\newlistof{math}{mathlist}{\listofmathcall}
% add parentheses around argument
\newcommand{\parent}[1]{ \ifx&#1&\else (#1) \fi }
% unnumerated mathematical definition environment definiton
\newenvironment{mathdef*}[2]{
	\begin{framed}close
	\noindent
	{ \fontfamily{ppl}\selectfont \textbf{\textsc{#1:}} } ~ #2 
	\par \hfill\\ 
	\fontfamily{lmr}\selectfont \itshape
}{
	\end{framed}
}
% definitions for numerated mathematical definition environment
\newcounter{mathdefc}[section]
\newcommand*{\mathdefnum}{\thesection.\arabic{mathdefc}}
\renewcommand{\themathdefc}{\mathdefnum}
\newenvironment{mathdef}[2]{
	\refstepcounter{mathdefc}
	\addcontentsline{mathlist}{figure}{\protect{\numberline{\mathdefnum}#1 ~ #2}}
	\begin{mathdef*}{#1 \mathdefnum}{#2}
}{
	\end{mathdef*}
}
% standard mathdef calls
\newcommand{\definitioncall}{Definition}
\newenvironment{stddef*}[1][]{ \begin{mathdef*}{\definitioncall}{\parent{#1}} }{ \end{mathdef*} }
\newenvironment{stddef}[1][]{ \begin{mathdef}{\definitioncall}{\parent{#1}} }{ \end{mathdef} }
% unnumerated theorem environment definition
\newenvironment{maththeorem*}[2]{
	\begin{leftbar}
	\noindent
	{ \fontfamily{ppl}\selectfont \textbf{\textsc{#1:}} } ~ #2
	\par \hfill\\ 
	\fontfamily{lmr} \fontshape{it} \selectfont
}{ 
	\end{leftbar}
}
% definitions for numerated theorem environment
\newcounter{maththeoremc}[section]
\newcommand*\maththeoremnum{\thesection.\arabic{maththeoremc}}
\renewcommand{\themaththeoremc}{\maththeoremnum}
\newenvironment{maththeorem}[2]{09.10.2015
	\refstepcounter{maththeoremc}
	\addcontentsline{mathlist}{figure}{\protect{\qquad\numberline{\maththeoremnum}#1 ~ #2}}
	\begin{maththeorem*}{#1 \maththeoremnum}{#2}
}{
	\end{maththeorem*}
}
% standard maththeorem calls
\newcommand{\theoremcall}{Theorem}
\newenvironment{theorem*}[1][]{ \begin{maththeorem*}{\theoremcall}{\parent{#1}} }{ \end{maththeorem*} }
\newenvironment{theorem}[1][]{ \begin{maththeorem}{\theoremcall}{\parent{#1}} }{ \end{maththeorem} }
\newcommand{\lemmacall}{Lemma}
\newenvironment{lemma*}[1][]{ \begin{maththeorem*}{\lemmacall}{\parent{#1}} }{ \end{maththeorem*} }
\newenvironment{lemma}[1][]{ \begin{maththeorem}{\lemmacall}{\parent{#1}} }{ \end{maththeorem} }
\newcommand{\propositioncall}{Proposition}
\newenvironment{proposition*}[1][]{ \begin{maththeorem*}{\propositioncall}{\parent{#1}} }{ \end{maththeorem*} }
\newenvironment{proposition}[1][]{ \begin{maththeorem}{\propositioncall}{\parent{#1}} }{ \end{maththeorem} }
\newcommand{\corollarycall}{Korollar}
\newenvironment{corollary*}[1][]{ \begin{maththeorem*}{\corollarycall}{\parent{#1}} }{ \end{maththeorem*} }
\newenvironment{corollary}[1][]{ \begin{maththeorem}{\corollarycall}{\parent{#1}} }{ \end{maththeorem} }
% q.e.d. definition
\newcommand{\qed}{ \par \hfill \fontfamily{lmr} \fontshape{it} \selectfont \mbox{q.e.d.} \\}
\newcommand{\qedbox}{ \par \hfill $\Box$ \\ }
% proof environment definition for theorems
\newenvironment{mathproof}[1]{
	\par \hfill\\
	\noindent
	{ \fontfamily{lmr}\selectfont \textsc{#1:}}
	\normalfont
	\small
	\begin{adjustwidth}{2em}{} 
}{ 
	\end{adjustwidth} 
	\qed
}
% standard mathproof calls
\newcommand{\proofcall}{Beweis}
\newenvironment{proof}{ \begin{mathproof}{\proofcall} }{ \end{mathproof} }
\newcommand{\proofideacall}{Beweisidee}
\newenvironment{proofidea}{ \begin{mathproof}{\proofideacall} }{ \end{mathproof} }

% new displaymath command, so that equations will not be stretched
\newcommand{\D}[1]{\mbox{$ #1 $}}
% make unnumerated equation
\newcommand{\E}[1]{\[ #1 \]}
% command for curly brackets
\newcommand{\curlb}[1]{\left\{ #1 \right\}}
% command for box brackets
\newcommand{\boxb}[1]{\left[ #1 \right]}
% command for parentheses/curved brackets
\newcommand{\curvb}[1]{\left( #1 \right)}
% command for angle brackets
\newcommand{\angleb}[1]{\left\langle #1 \right\rangle}
% command for floor brackets
\newcommand{\floorb}[1]{\left\lfloor #1 \right\rfloor}
% command for ceil brackets
\newcommand{\ceilb}[1]{\left\lceil #1 \right\rceil}
% command for creating sets
\newcommand{\set}[2]{ \left\{ #1 \enspace \middle\vert \enspace #2 \right\} }
% command for absolute value
\newcommand{\abs}[1]{\left\vert #1 \right\vert}
\newcommand{\norm}[1]{\left\| #1 \right\|}
% command for differential
\newcommand{\diff}{\mathrm{d}}
\newcommand{\Diff}{\mathrm{D}}
% command for derivative
\newcommand{\Deriv}[3][]{\Diff_{#2}^{#1}#3}
\newcommand{\deriv}[3][]{\dfrac{\diff^{#1}#2(#3)}{\diff #3^{#1}}}
% command for integral
\newcommand{\integral}[4]{\int_{#1}^{#2} #3\,\diff #4}
\newcommand{\Integral}[4]{\int\limits_{#1}^{#2} #3\,\diff #4}
% mathematical definitions (standard sets)
\newcommand{\SR}{\mathbbm{R}}
\newcommand{\SRP}{\SR^+}
\newcommand{\SRPN}{\SRP_0}
\newcommand{\SN}{\mathbbm{N}}
\newcommand{\SNN}{\SN_0}
\newcommand{\SZ}{\mathbbm{Z}}
\newcommand{\SQ}{\mathbbm{Q}}
\newcommand{\SQP}{\SQ^+}
\newcommand{\SQPN}{\SQP_0}

% command for physical units
\newcommand{\unit}[1]{\, \text{#1}}


% package for init listings(non-formatted  text) e.g. different source codes
\usepackage{listings}


% definitions for listing colors
\definecolor{codeDarkGray}{gray}{0.2}
\definecolor{codeGray}{gray}{0.4}
\definecolor{codeLightGray}{gray}{0.9}
\definecolor{codeBorder}{rgb}{0.34,0.24,0.21}
% predefinitions for listings
\newcommand{\listingcall}{Listing}
\newlength{\listingframemargin}
\setlength{\listingframemargin}{1em}
\newlength{\listingmargin}
\setlength{\listingmargin}{0.1\textwidth}
% \newlength{\listingwidth}
% \setlength{\listingwidth}{ ( \textwidth - \listingmargin * \real{2} + \listingframemargin * \real{2} ) }
% definitions for list of listings
\newcommand{\listoflistingscall}{\listingcall -Verzeichnis}
\newlistof{listings}{listinglist}{\listoflistingscall}
% style definition for standard code listings
\lstdefinestyle{std}{
	belowcaptionskip=0.5\baselineskip,
	breaklines=true,
	frameround=tttt,
	% frame=false,
	xleftmargin=0em,
	xrightmargin=0em,
	showstringspaces=false,
	showtabs=false,
	% tab=\smash{\rule[-.2\baselineskip]{.4pt}{\baselineskip}\kern.5em},
	basicstyle= \fontfamily{pcr}\selectfont\footnotesize\bfseries,
	keywordstyle= \bfseries\color{MidnightBlue}, %\color{codeDarkGray},
	commentstyle= \itshape\color{codeGray},
	identifierstyle=\color{codeDarkGray},
	stringstyle=\color{BurntOrange}, %\color{codeDarkGray},
	numberstyle=\tiny\ttfamily,
	% numbers=left,
	numbersep = 1em,
	% stepnumber = 1,
	% captionpos=t,
	tabsize=2,
	% backgroundcolor=\color{codebLightGray},
	rulecolor=\color{codeBorder},
	framexleftmargin=\listingframemargin,
	framexrightmargin=\listingframemargin
}
% definition for unnumerated listing
\newcommand{\inputlistingn}[3][]{
	\begin{center}
		\begin{adjustwidth}{\listingmargin}{\listingmargin}
			\centerline{ {\fontfamily{lmr}\selectfont\scshape \listingcall:}\quad #2 }
			\lstinputlisting[style=std, #1]{#3}
		\end{adjustwidth}
	\end{center}
}
% definition for numerated listing
\newcounter{listingc}[section]
\newcommand*\listingnum{\thesection.\arabic{listingc}}
\renewcommand{\thelistingc}{\listingnum}
\newcommand{\inputlisting}[3][]{
	\refstepcounter{listingc}
	\addcontentsline{listinglist}{figure}{\protect{\numberline{\listingnum:} #2 } }
	\inputlistingn[#1]{#2}{#3}
}


% package for including csv-tables from file
% \usepackage{csvsimple}
% package for creating, loading and manipulating databases
\usepackage{datatool}

% package for converting eps-files to pdf-files and then include them
\usepackage{epstopdf}
% use another program (ps2pdf) for converting
% !!! important: set shell_escape=1 in /etc/texmf/texmf.cnf (Linux/Ubuntu 12.04) for allowing to use other programs
% !!!			or use the command line with -shell-escape
\epstopdfDeclareGraphicsRule{.eps}{pdf}{.pdf}{
ps2pdf -dEPSCrop #1 \OutputFile
}


% package for reference to last page (output number of last page)
\usepackage{lastpage}
% package for using header and footer
% options: automate terms of right and left marks
\usepackage[automark]{scrpage2}
% \setlength{\headheight}{4\baselineskip}
% set style for footer and header
\pagestyle{scrheadings}
% \pagestyle{headings}
% clear pagestyle for redefining
\clearscrheadfoot
% set header and footer: use <xx>head/foot[]{Text} (i...inner, o...outer, c...center, o...odd, e...even, l...left, r...right)
\cfoot[]{\newline\newline\newline\pagemark}
% use that for mark to last page: \pageref{LastPage}
% set header separation line
\setheadsepline[\textwidth]{0.5pt}
% set foot separation line
\setfootsepline[\textwidth]{0.5pt}

\usepackage{tcolorbox}
\tcbuselibrary{many}
\tcbset{fonttitle=\footnotesize}

\usepackage{url}

\usepackage[numbers,square]{natbib}
\bibliographystyle{abbrvnat}

\usepackage{placeins}

\usepackage{subfigure}