\section{Zusammenfassung} % (fold)
\label{sec:zusammenfassung}

	Im Versuch sollten die Analyse von Kristallstrukturen mittels des Drehkristallverfahrens erlernt und angewendet werden.
	Dazu wurden Aufnahmen in den drei Richtungen der $(100)$-, $(110)$- und $(111)$-Achsen angefertigt und vermessen.
	Aus den entstandenen Reflexen konnten nun mittels beugungstheoretischer Grundlagen sowohl Gitterkonstanten als auch die konkrete Kristallstruktur ermittelt werden.
	Die Röntgenfilme zeigten zunächst die erwarteten Eigenschaften.
	Alle Reflexe ließen sich einer konkreten Schichtlinie zuordnen und ihr Abstand bestätigte die theoretischen Formeln.
	Die Auswertung ergab eine fcc-Struktur mit der Gitterkonstanten von LiF.
	Bei genauerer Betrachtung zeigten sich kleine Abweichungen beziehungsweise das mehrfache Auftreten von Reflexen.
	Dies deutet auf leichte Abweichungen in der Kristallstruktur sowie kleinere Fehler bei der Justage der Drehachse hin.
	Um diese zu beheben könnte man die Kristalle kleiner wählen und eventuell eine Nachjustierung anhand der ersten Röntgenaufnahme vornehmen.
	Eine etwas längere Belichtungszeit könnte eventuell auch noch mehr Reflexe hervorbringen beziehungsweise die sehr undeutlichen und schwach ausgeprägten stärker hervorheben.
	Dennoch deckten sich die gemessen Beugungsbilder gut mit den Erwartungen und die vorausgesagten Effekte wie Bragg-Reflexion und Einfluss des Strukturfaktors konnten nachgewiesen werden.
	Die Gegenüberstellung der Aufnahmen mit der Ag-Anode zeigte die Vorzüge eines Metallfilters, da hier die Aufnahmen noch stark mit Bremsstrahlung überlagert waren.
	Durch die veränderten Schichtlinienabstände konnte auf die kleinere Wellenlänge der $K_\alpha$-Linie von Silber in Vergleich zu Kupfer geschlossen werden.
	Auch konnten aus dem Intensitätsverlauf die Abstrahlcharakteristik der Ag-Anode qualitativ nachvollzogen werden.

	Zusammenfassend kann man sagen, dass das Drehkristallverfahren eine einfache und effektive Methode der Festkörperanalyse darstellt, die bei exakter Justage präzise Ergebnisse über Aufbau und Struktur liefern kann.
% section zusammenfassung (end)
