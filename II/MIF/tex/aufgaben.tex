\section{Aufgaben} % (fold)
\label{sec:aufgaben}

Ziele für das Michelson-Interferometer:

	\begin{itemize}
		\item Aufbau und Justierung des Michelson-Interferometers

		\item Erzeugung und Aufnahme von Interferenzen gleicher Dicke (Fizeau-Streifen) sowie gleicher Neigung (Haidinger Ringe)

		\item Einbringen einer zusätlichen Glasplatte, Untersuchung und Aufnahme der Wirkung auf das Interferenzmuster.
		Eigenschaften der Platte ableiten

		\item Wirkung von inhomogenen Brechungsindex durch erwärmte Luft ermitteln
	\end{itemize}

	Ziele für das MIF als Fourier-Spektrometer:

	\begin{itemize}
		\item Aufnahme eines Ein- und Zweilinien Interferogramms, Ermittlung des Spektrums und Untersuchung der Abhängigkeit der Auflösung von verschiedenen Messparametern

		\item Einfluss der Spiegelverschiebung auf Interferogramm des HeNe-Lasers studieren

		\item Fourier-Spektren der Hg-Lampe mit verschiedenen Filtern und der Na-Lampe vermessen

		\item Spektrum einer Glühlampe in Abhängigkeit des Lampenstroms und des verwendendeten Lichtsensors aufzeichnen

		\item Untersuchung der Absorbtionseigenschaften von Wasser und Benzol
		
	\end{itemize}

% section aufgaben (end)