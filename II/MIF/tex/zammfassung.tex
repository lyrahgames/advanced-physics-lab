% So ihr Affenärsche, jetzt wird zusammengefasst

\section{Zusammenfassung} % (fold)
\label{sec:zusammenfassung}

	Im Versuch wurde das Michelson-Interferometer als optisches Instrument zur präzisen Vermessung von Bauteilen und Lichtquellen, sowie als Grundlage der Fourier-Spektroskopie vorgestellt.
	Der erste Versuchsteil widmete sich dem Aufbau und den verschiedenen Interferenzerscheiungen, die am MIF erzeugt werden können, sowie deren Anwendungen.
	Es zeigte sich in beeindruckender Weise die hohe Empfindlichkeit der Interferenzbilder auf kleinste Veränderungen wie leichter Fingerdruck gegen einen der Spiegel oder bloßes Hauchen in den Strahlengang.
	Die typischen Interferenzbilder Fizeau-Streifen und Haidinger-Ringe konnten eingestellt werden und anhand ihrer Verformung ließen sich Rückschlüsse auf Abweichungen der Spiegel im $\,\mu\m{m}$-Bereich ziehen.
	Mit etwas Fingerspitzengefühle konnte sogar weißes Licht einer Glühlampe zur Interferenz gebracht werden und seine Kohärenzlänge zu etwa $4 \,\mu\m{m}$ bestimmt werden, was sich gut mit den Werten aus der Literatur deckt.
	Die Anwendung des MIF zur Vermessung optischer Bauteile konnte an einer Glasplatte begreiflich gemacht werden.
	Auch hier ließen sich mikrometergenaue Aussagen über die Dicke mit einfachen Überlegungen treffen.

	Der zweite Versuchsteil beleuchtete das Anwendungsgebiet des MIF in der Fourier-Spektroskopie.
	Auch hier konnten in relativ einfacher Weise erstaunlich genaue Aussagen über die Lage der Spektrallinien mehrerer Lichtquellen getroffen werden.
	Es konnten sehr gute Auflösungen im Bereich von $10^{3} - 10^{4}$ erreicht werden und somit ließen sich alle Doppelpeaks und sogar Lasermoden trennen und einzeln detektieren.
	Die angezeigten Interferogramme deckten sich mit den Erwartungen über Ein- und Zwei-Linieninterferenzen.
	Aufgrund der endlichen Aufnahme Zeit und den damit verbundenen \glqq Abbruchkanten \grqq\ sowie dem leichten Ruckeln entstanden an den Rändern der Spektren immer wieder Artefakte wie sehr hohe Frequenzpeaks, die jedoch durch richtige Wahl des betrachteten Ausschnitts ausgeblendet werden konnten.
	Bei Aufnahme der Glühemissionsspektren zeigte sich die Notwendigkeit von Detektoren mit geringe Bandlücke wie Bleiselenid um auch noch langwellige Infrarotstrahlung messen zu können.
	Auch bei der Untersuchung von Wasser stimmten die erhobenen Messdaten gut mit den Erwartungen des Absorptionsverhalten überein.

	Zusammenfassend kann man sagen, dass das Michelson-Interferometer ein hoch präzises Messinstrument zum hoch präzisen messen darstellt. Es ist wirklich sehr sehr präzise. Und hoch. (Das muss überarbeitet werden.)
	
	Zusammenfassend kann man sagen, dass das Michelson-Interferometer ein hoch präzises optisches Instrument darstellt, dessen Anwendungsmöglichkeiten von Vermessung optischer Bauteile auf Bruchteile von Wellenlängen bis zum Nachweis von Gravitationswellen reicht.
	Speziell in der Spektroskopie liefert das MIF gerade im langwelligen Bereich gute Werte und kann durch seine Arbeitsweise in Gegensatz zu z.B. Gitterspektrografen auch mit sehr geringen Intensitäten arbeiten.
	Durch genau Einstellungen des Motors, genügend hohe Samplezahl und lange Verschiebung können so heute leicht Auflösungen von zu $10^{6}$ erreicht werden.
	Wenn das nicht geil ist.
	  

% section zusammenfassung (end) ersten 