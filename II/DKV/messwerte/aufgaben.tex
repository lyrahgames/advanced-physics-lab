\section{Aufgaben} % (fold)
\label{sec:aufgaben}
	
	Bestimmen der Gitterkonstanten eines kubischen Kristalls mit Hilfe des Drehkristallverfahrens.

	\begin{itemize}
		\item Drehkristallaufnahmen mit den Drehachsen [100], [110] und [111] und Bestimmung des Bravaisgittertyps durch Auswertung der Schichtlinienabstände 
		\item Indizierung der Reflexe (insbesondere der nullten Schichtlinie) 
		\item Bestimmung der Gitterkonstanten sowohl aus den Schichtlinienabständen als auch aus den Reflexpositionen in der nullten Schichtlinie 
		\item Identifizieren Sie durch Zuordnung der Gitterkonstanten die von Ihnen untersuchte Substanz. 
		\item Ausführliche Fehlerdiskussion der Gitterkonstanten. Berechnen Sie in diesem Zusammenhang die maximale Probenabsorption und diskutieren Sie auch deren Einfluss auf die Genauigkeit der Gitterkonstantenbestimmung. 
		\item Anfertigen einer Aufnahme ohne Absorptionsfilter und Interpretation des Resultats. 
		% \item Werten Sie das von Ihnen getestete Röntgenverfahren, das in der Literatur als „Drehkristall“ bekannt ist (wobei auch bei anderen Röntgenverfahren die Drehung bzw. Schwenkung der Probe als selbstverständlich gilt). 
	\end{itemize}
% section aufgaben (end)