\section{Aufgaben}
\label{sec:aufgaben}
	
	\begin{enumerate}
		\item Abschätzung einer oberen Strahlendosis.
			\begin{itemize}
				\item Schätzen Sie eine obere Grenze der zusätzlichen Äquivalentdosis ab.
			\end{itemize}
		\item Messung und Interpretation eines $\gamma$-Spektrums
			\begin{itemize}
				\item Stellen Sie im Vielkanalanalysator das mit dem Ge-Detektor gemessene Spektrum der $\prescript{22}{}{\m{Na}}$-Quelle dar.
				Kennzeichnen und erklären Sie alle erkennbaren Details.
				\item Skizzieren Sie das Ausgangssignal des Verstärkers und kennzeichnen Sie die erkennbaren Details, die sich im Spektrum wiederfinden.
			\end{itemize}
		\item Energieauflösung des Ge-Detektors
			\begin{itemize}
				\item Nehmen Sie das $\gamma$-Spektrum von $\prescript{152}{}{\m{Eu}}$ im Abstand von $30\unit{cm}$ auf und bestimmen Sie die Breite der markantesten Linie im Energiebereich zwischen $120\unit{keV}$ und $1\unit{Mev}$ durch Anpassung der Linien mit einer Gaußfunktion.
				\item Tragen Sie die relative Energieauflösung als Funktion der Energie im Bereich $120\unit{keV}$ bis $1\unit{MeV}$ auf und diskutieren Sie Ihr Ergebnis.
				\item Bestimmen Sie die die durchschnittliche Arbeit zur Erzeugung eines Elektron-Loch-Paares. Vergleichen Sie diese mit der Bandlücke von Germanium und diskutieren Sie ihr Ergebnis.
			\end{itemize}
		\item Totzeitkorrektur
			\begin{itemize}
				\item Nehmen Sie das $\gamma$-Spektrum von $\prescript{152}{}{\m{Eu}}$ für verschiedene Abstände auf und notieren Sie sich die Totzeiten.
				\item Ermitteln Sie mithilfe einer Störquelle Korrekturfaktoren.
				\item Tragen Sie die auf Totzeit korrigierten Zählraten für $122\unit{keV}$ und $1408\unit{keV}$ für die verschiedenen Abstände geeignet auf, um durch die Anpassung eine Gerade zu bestimmen.
			\end{itemize}
		\item Nachweiswahrscheinlichkeit des Ge-Detektors
			\begin{itemize}
				\item Ermitteln Sie die Nachweiswahrscheinlichkeit als Funktion der $\gamma$-Energie.
			\end{itemize}
		\item Identifizierung und Berechnung der Aktivität einer unbekannten Quelle
			\begin{itemize}
				\item Führen Sie eine Energiekalibrierung durch, die im Bereich bis $200\unit{keV}$ besonders genau ist.
				Nehmen Sie das Spektrum der unbekannten Quelle bei einem gegebenen Abstand auf.
				\item Identifizieren Sie das Isotop.
				\item Bestimmen Sie die Aktivität dieser Quelle.
			\end{itemize}
		\item Vergleich von $\gamma$-Detektoren
			\begin{itemize}
				\item Vergleichen Sie die $\gamma$-Spektren von $\prescript{22}{}{\m{Na}}$ für den HP-Ge-Detektor, den NaI(Tl)- und den Plastikszintillationsdetektor.
				Stellen Sie die drei Spektren übereinander dar.
				Warum fehlen die Photopeaks beim beim Plastikszintillationsdetektor?
				\item Bestimmen Sie die Energieauflösung bei $1332\unit{keV}$.
				Warum ist die Energieauflösung des Ge-Detektors besser.
				\item Welche Ereignisse bestimmen maßgeblich die Energieauflösung beim Plastikszintillationsdetektor?
			\end{itemize}
	\end{enumerate}


% section aufgaben