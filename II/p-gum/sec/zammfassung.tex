\section{Zusammenfassung}
\label{sec:zammfassung}
	
	Im Experiment sollten die Grundlagen der $\gamma$-Spektroskopie erlernt und der Umgang mit wichtigen Detektortypen wie Halbleiter- oder Szintillationsdetektoren geübt werden.
	Die Aufnahmen der $\prescript{22}{}{\m{Na}}$-, $\prescript{152}{}{\m{Eu}}$- und $\prescript{137}{}{\m{Cs}}$-Quellen zeigten alle erwarteten $\gamma$-Linien wie sie auch in Literatur zu finden sind.
	Hier zeigte sich auch die große Notwendigkeit einer gründlichen Kalibrierung, mit deren Hilfe dann auch sehr gute quantitative Ergebnisse erzielt wurden.
	Viele der in den Grundlagen vorhergesagten Effekte wie Positronenvernichtung, Comptonkante, Comptonuntergrund und Rückstoßpeak fanden sich in den Messungen wieder.
	Im natürlichen Spektrum konnten Spuren der Elemente aus den natürlichen Zerfallsreihen wie etwa $\prescript{40}{}{\m{K}}$ nachgewiesen werden, was zeigt, wie empfindliche die Methode auf selbst sehr geringe Konzentrationen und Aktivitäten reagiert.
	Durch messtechnische Korrekturen wie Normierung auf effektive Aufnahmezeit und Totzeitkorrektur konnte dass $1/r^2$-Gesetz bestätigt werden, wenn auch das Offset $d_0$ einen größeren Wert als erwartet annahm.
	Beim Vergleich der Detektortypen zeigte sich eindeutig die wesentlich höhere Auflösung von Halbleiterdetektoren wie HP-Ge gegenüber Szintillationsdetektoren wie NaI(Tl).
	Das fehlen der Comptonkannten beim Plastik-Detektor deckte sich mit den Erwartungen aus den Grundlagen.
	Die unterschiedlichen Auflösungen liegen an der sehr unterschiedlichen ausgelößten Teilchenzahl (seien es Ladungsträger oder Photonen), was die theoretischen Überlegungen zum Zusammenhang zwischen Linienbreite $\sigma$ und Teilchenzahl $N_T$ bestätigt.
	Auch die Annahme über die gaußförmige Linienverbreiterung konnte durch entsprechende Fits verifiziert werden.

	Zusammenfassend kann man also sagen, dass die $\gamma$-Spektroskopie mittels HP-Ge-Detektor eine sehr effiziente Methode zur Untersuchung von radioaktiver Strahlung darstellt, die über einen weiten Energiebereich bei richtiger Kalibrierung hoch aufgelöste Ergebnisse erzielen kann.
	Mit ihrer Hilfe ließen sich sowohl qualitative als auch quantitative Aussagen über die Zusammensetzung der verwendeten Quellen wie auch der natürlichen Strahler anstellen.
	Die Szintillationsdetektoren ließen auch Rückschlüsse über die eingehende Strahlung zu, besaßen aber bei weitem nicht das Auflösungsvermögen des HP-GE-Detektors.
	Im Gegenzug besitzen sie weniger Ansprüche an Material und Kühlung und können auch ausgedehnte Quellen vermessen.


% section zammfassung