\section{Zusammenfassung} % (fold)
\label{sec:zusammenfassung}

	% Zusammenfassend kann man sagen, wenn ihr vor der Wahl steht diesen Versuch zu machen oder euch mit Werwölfen, dem Joker, Smaug und der kompletten wilden Jagd über das Wochenende in eine gemütliche Waldhütte reinzuteilen, dann überlegt gut. 
	% Bei den anderen gibt es zumindest eine Chance es unbeschadet zu überstehen. 
	% Klein aber immerhin.

	% OK, jetzt ordentlich:
	Im Experiment wurde das Verhalten von supraleitenden Materialien unter den verschiedenen Einflüssen untersucht. 
	Typische Eigenschaften wie die schlagartige Widerstandsabnahme bei Unterschreitung der  Sprungtemperatur oder weitere Verringerung der selben durch äußere Magnetfelder konnten beobachtet und dokumentiert werden. 
	Auch der Übergang zum ohmschen Verhalten bei Überschreiung des kritischen Stromes war beim SIS-Josephson-Kontakt zu sehen. 
	Der 
	% pervertierte äh...
	invertierte Josephson-Effekt beim Einkoppeln von Mikrowellen war auf Grund des zu labilen Versuchsaufbaus 
	% (labil ist die kleine Schwester von SCHEIßE) 
	leider nicht vollständig zu sehen. 
	Hier würden sich wie auch beim letzten Versuchsteil, der SQUID-Messung, vorgefertigte, definierte und widerstandsfähigere Kontakte eignen. 
	Dennoch konnte mit Hilfe der gereuzten Drähte ein SQUID erstellt werden und seine Reaktion auf das Magnetfeld entsprach den Vorhersagen. 
	Die über Magnetfeld und Spannungsverlauf abgeschätzte SQUID-Größe liegt von der Größenordnung gut im erwarteten Bereich.
	Zusammenfassend kann man sagen, dass das Phänomen der Supraleitung viele interessante und unerwartete Effekte mit sich bringt, die in jüngster Vergangenheit und wohl auch in Zukunft zu großen Innovationen führten und führen werden. Ungeahnte Messgenauigkeit, verlustfreie Energieübertragung, Magnetschwebebahnen und ultra schnelle Prozessoren sind nur einige der
	%voll krass, hammergeilen
	Möglichkeiten, die die Supraleitungstechnologie ermöglichen könnte.

	% The Futur has begun!!!
	% Die Macht ist stark in der Supraleitung!
	% The World needs no Hero. The World needs a working SQUID.
	% Ein kleiner Stoß für einen Menschen, jedoch ein Messwerte zerstörender Stoß für ein Supraleitungsexperiment.
	% Das vor ihnen liegende Protokoll ist ein Werk der Fiktion. Und des mangelnden Bocks.
	% Beim Erstellen dieses Protokolls sind keine Tiere zu schaden gekommen.

% section zusammenfassung (end)

