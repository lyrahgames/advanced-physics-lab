Im Versuch sollten sich intensiv mit den Möglichkeiten und Anforderungen der Auger-Elektronen-Spektroskopie auseinandergesetzt werden.
Es zeigte sich, dass die AES ein sehr empfindliches und auch aufwendiges Verfahren zur Oberflächenanalyse darstellt, dass bei langer Messzeit, optimalen Parametern und hoher Sauberkeit der Probe auch sehr kleine Konzentrationen von Elementen auf der Oberfläche nachweisen kann. Die erwarteten Peaks von Silber, Sauerstoff, Kohlenstoff und Aluminium stimmten alle in geringen Toleranzbereichen mit den Tabellenwerten überein. 
Abweichungen ergaben sich vor allem in der Form und erwarteten Intensität der Peaks.
So zeigte Sauerstoff oft eine größere Asymmetrie auf, während sie bei Kohlenstoff fehlte. Auch der Doppelausschlag bei Silber zeigte in der Mitte nicht die erwartete Intensität. Dennoch muss man sagen, das im qualitativen Bereich gute Analysen der jeweiligen Proben vorgenommen werden konnten.
Die qualitative Analyse war aufgrund der oft von einer auf die nächste Messung wechselnden Intensitäten und unklaren Peak Abgrenzungen eher schwierig und uneindeutig. Hierfür sind längere Messungen und größere Probenströme notwendig.
Überraschend war das hohe Vorkommen von Kohlenstoff auf fast jeder Probe. Aus diesem Grund halten wir Sputtern vor jeder genaueren Spektroskopie für notwendig.