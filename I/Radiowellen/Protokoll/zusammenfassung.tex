Im Versuch sollten die elektrischen Eigenschaften von Leitungen im Hochfrequenzbereich ermittelt werden. 
Desweiteren sollten typische Verfahren der analogen Signalverarbeitung und Übertragung wie Amplitudenmodulation und die Emission und Ausbreitung elektromagnetischer Wellen im MHz-Bereich untersucht werden.
Beim Ausmessen der Kabel zeigte sich die unbedingte Notwendigeit der Anpassung in der Hochfrequenztechnik.
Bereits kleine Abweichungen des Wellenwiderstandes vom Eingangswiderstand des jeweiligen Verbrauchers verursachten direkt die Ausbildung stehender Wellen, die, da die Wellenlänge im untersuchten Bereich von 0 bis ca. 300 MHz gut mit der Leiterlänge übereinkam, zu starken Verfälschungen der Übertragungsfunktion führten. 
So konnte ein einfaches Stichkabel ohne Abschluss mit Wellenwiderstand als Frequenzfilter wirken und betimmte Frequenzen fast komplett auslöschen.
Die Messungen der Lichtgeschwindigkeiten und Verkürzungsfaktoren im Kabelmedium bestätigten gut die theoretischen Annahmen zur Ausbreitung elektromagnetischer Wellen im Metall.

Die bei der additiven Amplitudenmodulation eingesetzten Schaltungen lieferten die erwarteten Signale, es konnten Träger und Seitenbänder im Spektrum ermittelt und quantitativ zugeordnet werden.
Zusammen mit den experimentell ermittelten Kenntnissen über ideale Abstrahlfrequenzen von $\lambda /2$-Antennen war es uns möglich, ein willkürliches niederfrequentes Signal wie z.B. Musik zu modulieren, abzustrahlen, zu empfangen, zu demodulieren und schließlich über Lautsprecher wiederzugeben. 
In Anbetracht der relativ einfachen Schaltungspläne und Geräte, war die empfangene Qualität des Signals erstaunlich gut.
Dies zeigt die ungemeinen Vorteile und Möglichkeiten der Signalübertragung mittels hochfrequenter modulierter elektromagnetischer Wellen.